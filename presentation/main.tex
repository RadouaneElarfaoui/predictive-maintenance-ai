\documentclass{beamer}

\usetheme{Madrid}
\usecolortheme{whale}

\usepackage[utf8]{inputenc}
\usepackage[T1]{fontenc}
\usepackage[french]{babel}
\usepackage{graphicx}
\usepackage{booktabs}
\usepackage{enumerate}

% Meta-data
\title[Maintenance Prédictive]{Application de l'IA en Mécanique : Maintenance Prédictive}
\subtitle{Projet de fin d'étude}
\author{Étudiant}
\institute[]{Département de Génie Mécanique}
\date{\today}

\begin{document}

% Title Page
\begin{frame}
    \titlepage
\end{frame}

% Table of Contents
\begin{frame}{Plan de la présentation}
    \tableofcontents
\end{frame}

% Section 1: Introduction
\section{Introduction}
\begin{frame}{Introduction}
    \begin{itemize}
        \item \textbf{Contexte :} Convergence de l'ingénierie mécanique et de l'intelligence artificielle (Industrie 4.0).
        \item \textbf{Sujet :} Maintenance prédictive des machines industrielles.
        \item \textbf{Objectif :} Anticiper les pannes pour optimiser la production et réduire les coûts.
        \item \textbf{Approche :} Utilisation de réseaux de neurones artificiels (RNA) sous MATLAB.
    \end{itemize}
\end{frame}

% Section 2: Contexte et Objectifs
\section{Contexte et Objectifs}
\begin{frame}{Problématique}
    Les approches traditionnelles de maintenance présentent des limites :
    \vspace{0.5cm}
    \begin{itemize}
        \item \textbf{Maintenance Corrective :} 
        \begin{itemize}
            \item Réactive (après la panne).
            \item Coûts d'arrêt imprévisibles et élevés.
        \end{itemize}
        \item \textbf{Maintenance Préventive :}
        \begin{itemize}
            \item Planifiée (intervalles fixes).
            \item Gaspillage potentiel (remplacement de pièces saines).
        \end{itemize}
    \end{itemize}
    \vspace{0.5cm}
    \textbf{Question :} Comment intervenir au \textit{bon moment} ?
\end{frame}

\begin{frame}{Objectif du Projet}
    Développer un système de \textbf{Maintenance Prédictive Intelligent}.
    \vspace{0.5cm}
    \begin{block}{Mission}
        Concevoir et entraîner un modèle de réseau de neurones capable de classifier l'état de la machine en temps réel :
        \begin{itemize}
            \item \textbf{0} : Fonctionnement Normal
            \item \textbf{1} : Panne Détectée
        \end{itemize}
    \end{block}
    \vspace{0.5cm}
    Basé sur les données de capteurs (température, vitesse, couple, etc.).
\end{frame}

% Section 3: Méthodologie
\section{Méthodologie}
\begin{frame}{Description des Données}
    Données : \texttt{machine failure.csv} (Conditions de fonctionnement d'une machine-outil).
    
    \textbf{Variables d'entrée (Features) :}
    \begin{itemize}
        \item \textbf{Type de produit} (L, M, H) : Qualité/Contraintes.
        \item \textbf{Température de l'air [K]} : Ambiance.
        \item \textbf{Température du processus [K]} : Pièce/Outil.
        \item \textbf{Vitesse de rotation [rpm]} : Broche.
        \item \textbf{Couple [Nm]} : Force de torsion.
        \item \textbf{Usure de l'outil [min]} : Temps d'utilisation cumulé.
    \end{itemize}
    
    \textbf{Variable Cible (Target) :}
    \begin{itemize}
        \item \textbf{Machine Failure} (0 ou 1).
    \end{itemize}
\end{frame}

\begin{frame}{Implémentation sous MATLAB}
    \textbf{Outils :} MATLAB + Neural Network Toolbox.
    
    \vspace{0.3cm}
    \textbf{Étapes clés :}
    \begin{enumerate}
        \item \textbf{Préparation :} Chargement CSV, conversion 'Type' (catégoriel $\to$ numérique), transposition (Features $\times$ Samples).
        \item \textbf{Architecture :} \texttt{patternnet} (Reconnaissance de formes).
        \begin{itemize}
            \item 2 couches cachées : 100 et 50 neurones.
        \end{itemize}
        \item \textbf{Entraînement :} 
        \begin{itemize}
            \item Répartition : 70\% Train, 15\% Val, 15\% Test.
            \item Algorithme : Scaled Conjugate Gradient (trainscg).
        \end{itemize}
    \end{enumerate}
\end{frame}

\begin{frame}{Architecture du Réseau}
    \begin{figure}
        \centering
        \includegraphics[width=0.9\textwidth]{../rapport/src/NN_Training_Tool_Summary.jpg}
        \caption{Résumé de l'architecture et de l'entraînement (MATLAB)}
    \end{figure}
\end{frame}

% Section 4: Résultats
\section{Résultats et Analyse}
\begin{frame}{Performance Globale}
    \begin{alertblock}{Résultats sur l'ensemble de Test}
        Le modèle atteint une précision (\textit{Accuracy}) de \textbf{97.5\%}.
    \end{alertblock}
    
    \vspace{0.5cm}
    Analyse par classe :
    \begin{itemize}
        \item \textbf{Classe 0 (Normal) :} Excellente détection.
        \begin{itemize}
            \item Précision : 98.0\%
            \item Rappel : 99.4\%
        \end{itemize}
        \item \textbf{Classe 1 (Panne) :} Plus complexe (déséquilibre des classes).
        \begin{itemize}
            \item Précision : 69.2\%
            \item Rappel : 38.3\%
        \end{itemize}
    \end{itemize}
\end{frame}

\begin{frame}{Matrices de Confusion}
    \begin{figure}
        \centering
        \includegraphics[height=0.75\textheight]{../rapport/src/matlab.jpg}
        \caption{Matrices de confusion globale}
    \end{figure}
\end{frame}

\begin{frame}{Courbes ROC}
    La courbe ROC montre la capacité de discrimination du modèle.
    \begin{figure}
        \centering
        \includegraphics[height=0.65\textheight]{../rapport/src/ROC_Curves_Plot.jpg}
        \caption{Courbes ROC : Proches du coin supérieur gauche (Excellente performance).}
    \end{figure}
\end{frame}

\begin{frame}{Validation et Erreur}
    \begin{columns}
        \begin{column}{0.5\textwidth}
            \centering
            \includegraphics[width=\textwidth]{../rapport/src/MSE_Performance_Plot.jpg}
            \caption{Performance (Cross-Entropy)}
        \end{column}
        \begin{column}{0.5\textwidth}
            \centering
            \includegraphics[width=\textwidth]{../rapport/src/Error_Histogram_Plot.jpg}
            \caption{Histogramme des erreurs}
        \end{column}
    \end{columns}
    \vspace{0.2cm}
    \footnotesize Pas de sur-apprentissage majeur (arrêt précoce à l'époque 37).
\end{frame}

% Section 5: Conclusion 
\section{Conclusion}
\begin{frame}{Conclusion}
    \begin{itemize}
        \item \textbf{Réussite technique :} Mise en place d'un réseau de neurones performant (97.5\% de précision globale) avec MATLAB.
        \item \textbf{Maintenance Prédictive :} Preuve de concept validée pour l'anticipation des pannes à partir de données capteurs.
        \item \textbf{Perspectives :}
        \begin{itemize}
            \item Améliorer le rappel sur la classe "Panne" (gestion du déséquilibre des données).
            \item Tester d'autres architectures (Deep Learning, LSTM pour les séries temporelles).
            \item Déploiement sur un système temps réel.
        \end{itemize}
    \end{itemize}
\end{frame}

\begin{frame}
    \centering
    \Huge \textbf{Merci de votre attention}
    
    \vspace{1cm}
    \Large Avez-vous des questions ?
\end{frame}

\end{document}
